%%%%%%%%%%%%%%%%%%%%%%%%%%%%%%%%%%%%%%%%%%%%%%%%%%%%%%%%%%%%%%%%%%%%%%%%
% Plantilla para escribir libros
% Universidad de A Coruña. Facultad de Informática
% Realizado por: Welton Vieira dos Santos
% Modificado: Welton Vieira dos Santos
% Contacto: welton.dosssantos@udc.es
%%%%%%%%%%%%%%%%%%%%%%%%%%%%%%%%%%%%%%%%%%%%%%%%%%%%%%%%%%%%%%%%%%%%%%%%


\chapter*{Resumen}



\todo[inline]{}
	
ROS (Robot Operating System) es un set de algoritmos de código abierto, drivers, librerías y herramientas diseñadas y desarrolladas para la creación de controladores y aplicaciones para robots.

El objetivo de este trabajo es ofrecer una visión generalizada de que es ROS, donde se comenta un poco de su historia, que es ROS, su objetivo y además un poco de su evolución.


  




